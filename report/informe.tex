\documentclass{article}

% Paquetes plantilla
\usepackage[square, numbers, sort]{natbib}
% \usepackage{cite}
\usepackage{graphicx}
\usepackage[utf8]{inputenc}
\usepackage{amsmath,amssymb,amsfonts}
\usepackage{algorithmic}
\usepackage{textcomp}
\usepackage{subfig}
\usepackage{float}
\usepackage{mathtools}
\usepackage[spanish]{babel}
\usepackage[paper=a4paper,margin=2.75cm]{geometry}
\usepackage{booktabs} % for tables
\usepackage[colorlinks = true,
            linkcolor = blue,
            urlcolor  = blue,
            citecolor = orange,
            anchorcolor = blue]{hyperref} % for hyperlinks
\usepackage{xcolor} % text colors
% Paquetes extra
\usepackage{subcaption, booktabs, siunitx, tikz}
\usepackage{pgfplots}
\pgfplotsset{compat=1.15}
\usepackage{mathrsfs}
\usetikzlibrary{arrows}
\tikzset{every picture/.style={line width=0.75pt}} %set default line width to 0.75pt 

\sisetup{
    round-mode          = places, % Rounds numbers
    round-precision     = 2, % to 2 places
}

\usepackage{titlesec}

\setcounter{secnumdepth}{4}

\titleformat{\paragraph}
{\normalfont\normalsize\bfseries}{\theparagraph}{1em}{}
\titlespacing*{\paragraph}
{0pt}{3.25ex plus 1ex minus .2ex}{1.5ex plus .2ex}

% Directorio con imagenes
\graphicspath{{./figs/}}

% Cabecera del documento
% ======================================================================

% Titulo
\title{PROYECTO AUTOMATAS}

% Autores
\author{Juan Pablo Sibecas \\ juan.sibecas@gmail.com \\ Autómatas y Control Discreto, Facultad de Ingeniería, \\ Universidad Nacional de Cuyo, \\ Mendoza, Argentina}

% Fecha
\date{Febrero de 2024}

% Cuerpo del documento
% ======================================================================
\begin{document}

% Comandos definidos por el autor
\renewcommand{\tablename}{Tabla}
% \renewcommand{\color{blue}{#1}}{\azul}

% Crear cabecera
\maketitle

% Resumen
% ======================================================================
\begin{abstract}\label{sec:abstract}

\end{abstract}

\newpage

\section{Introducción} \label{sec:intro}

\section{Desarrollo} \label{sec:desarrollo}
    \subsection{Modelo del Sistema Físico} \label{sec:plantModel}

        \subsubsection{Modelo de Traslación Horizontal}

            \begin{equation} \label{eq:motor}
                J_m \dot{\omega_m(t)} = T_m(t) - T_l(t) - b_m \omega_m(t)
            \end{equation}

            comportamiento rueda:
            \begin{equation} \label{eq:wheel}
                J_w \dot{\omega_w(t)} = T_q(t) - T_w(t) - b_w \omega_w(t)
            \end{equation}

            relacion de transmision
            \begin{equation} \label{eq:transmission}
                r = \frac{\omega_m(t)}{\omega_w(t)} = \frac{T_q(t)}{T_l(t)}
            \end{equation}

            si reemplazo \ref{eq:transmission} en \ref{eq:motor} y despejo $T_q(t)$

            \begin{equation} \label{eq:Tq}
                T_q(t) = J_m \dot{\omega_w} r^2 + b_m \omega_w(t) r^2 + T_m(t) r
            \end{equation}

            reemplazando en \ref{eq:wheel} se obtiene

            \begin{equation} \label{eq:SSeq1}
                (J_w + J_m r^2) \dot{\omega_w}(t) = T_m(t) r - (b_w + b_m r^2) \omega_w (t) - T_w(t)
            \end{equation}

            como no hay resbalamiento, 

            \begin{align} \label{eq:trans2rot} %%REEMPLAZAR & POR FLECHA
                x_t(t) = R_w \theta_w(t) &\Rightarrow \theta_w(t) = \frac{x_t(t)}{R_w}\\
                \dot{x_t}(t) = R_w \omega_w(t) &\Rightarrow \omega_w(t) = \frac{\dot{x_t}(t)}{R_w}\\
                \ddot{x_t}(t) = R_w \dot{\omega_w}(t) &\Rightarrow \dot{\omega_w}(t) = \frac{\ddot{x_t}(t)}{R_w}\\
            \end{align}

            la fuerza que impulsa al carro es:
    
            \begin{equation} \label{eq:trolleyForce1}
                F_t(t) = \frac{T_w(t)}{R_w}
            \end{equation}

            reemplazando \ref{eq:trolleyForce1} y \ref{eq:trans2rot} en \ref{eq:SSeq1} se obtiene


            \begin{equation} \label{eq:trolleyForce2}
                F_t(t) = \frac{J_w + J_m r^2}{{R_w}^2} \ddot{x_t}(t) + \frac{r_t}{R_w} T_m(t) - \frac{b_w + b_m r^2}{{R_w}^2} \dot{x_t}(t)
            \end{equation}

            la traslación del carro, por segunda ley de Newton es:

            \begin{equation} \label{eq:trolleyMovement}
                m_t \ddot{x_t}(t) = F_t(t) + F_l(t) - b_t \dot{x_t}(t)
            \end{equation}
            
            $F_l(t)$ es la fuerza de la carga por balanceo:

            \begin{equation} \label{eq:cableForce}
                F_l(t) = F_w \sin{\theta(t)}
            \end{equation}

            Donde $F_w$ es la fuerza elástica del cable sobre el carro.

            reemplazando \ref{eq:cableForce}  y \ref{eq:trolleyForce2} en \ref{eq:trolleyMovement}


            \begin{equation} \label{eq:mechSSlong}
                (m_t + \frac{J_w + J_m r^2}{{R_w}^2}) \ddot{x_t}(t) = F_w \sin{\theta(t)} +  \frac{r_t}{R_w} T_m(t) - (b_t + \frac{b_w + b_m r^2}{{R_w}^2}) \dot{x_t}(t)
            \end{equation}

            que se puede expresar como:

            \begin{equation} \label{eq:mechSSshort}
                m_{eq} \ddot{x_t}(t) = F_w \sin{\theta(t)} +  \frac{r_t}{R_w} T_m(t) - b_{eq} \dot{x_t}(t)
            \end{equation}

            Donde

            \begin{equation} \label{eq:eqMass}
                m_{eq} = m_t + \frac{J_w + J_m r^2}{{R_w}^2}
            \end{equation}

            y

            \begin{equation} \label{eq:eqDamp}
                b_{eq} = b_t + \frac{b_w + b_m r^2}{{R_w}^2}
            \end{equation}


        \subsubsection{Modelo de Izaje}




\section{Resultados} \label{sec:results}

\section{Conclusión} \label{sec:conclusion}


% Estilo de citas
%\bibliographystyle{unsrt}

% Nombre del archivo .bib
%\bibliography{references}

\end{document}