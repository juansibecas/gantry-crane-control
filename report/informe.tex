\documentclass{article}

% Paquetes plantilla
\usepackage[square, numbers, sort]{natbib}
% \usepackage{cite}
\usepackage{graphicx}
\usepackage[utf8]{inputenc}
\usepackage{amsmath,amssymb,amsfonts}
\usepackage{algorithmic}
\usepackage{textcomp}
\usepackage{subfig}
\usepackage{float}
\usepackage{empheq}
\usepackage{mathtools}
\usepackage[spanish]{babel}
\usepackage[paper=a4paper,margin=2.75cm]{geometry}
\usepackage{booktabs} % for tables
\usepackage[colorlinks = true,
            linkcolor = blue,
            urlcolor  = blue,
            citecolor = orange,
            anchorcolor = blue]{hyperref} % for hyperlinks
\usepackage{xcolor} % text colors
% Paquetes extra
\usepackage{subcaption, booktabs, siunitx, tikz}
\usepackage{pgfplots}
\pgfplotsset{compat=1.15}
\usepackage{mathrsfs}
\usetikzlibrary{arrows}
\tikzset{every picture/.style={line width=0.75pt}} %set default line width to 0.75pt 

\sisetup{
    round-mode          = places, % Rounds numbers
    round-precision     = 2, % to 2 places
}

\usepackage{titlesec}

\setcounter{secnumdepth}{4}

\titleformat{\paragraph}
{\normalfont\normalsize\bfseries}{\theparagraph}{1em}{}
\titlespacing*{\paragraph}
{0pt}{3.25ex plus 1ex minus .2ex}{1.5ex plus .2ex}

% Directorio con imagenes
\graphicspath{{./figs/}}

% Cabecera del documento
% ======================================================================

% Titulo
\title{PROYECTO AUTOMATAS}

% Autores
\author{Juan Pablo Sibecas \\ juan.sibecas@gmail.com \\Matias Gaviño\\ matias.linares.g@gmail.com \\ Autómatas y Control Discreto, Facultad de Ingeniería, \\ Universidad Nacional de Cuyo, \\ Mendoza, Argentina}

% Fecha
\date{Junio de 2024}

% Cuerpo del documento
% ======================================================================
\begin{document}

% Comandos definidos por el autor
\renewcommand{\tablename}{Tabla}
% \renewcommand{\color{blue}{#1}}{\azul}

% Crear cabecera
\maketitle

% Resumen
% ======================================================================
\begin{abstract}\label{sec:abstract}

\end{abstract}

\newpage

\section{Introducción} \label{sec:intro}

\section{Desarrollo} \label{sec:desarrollo}
    \subsection{Modelo del Sistema Físico} \label{sec:plantModel}

        \subsubsection{Subsistema de Izaje}
            Segunda ley de Newton del lado tambor:
            \begin{equation} \label{eq:tamborIzaje}
                J_{hd+hEb} \frac{d \omega_{hd}}{dt} = T_{hd}(t) + T_{hEb}(t) - b_{hd} \omega_{hd}(t) - T_{hdl}(t)
            \end{equation}

            Segunda ley de Newton del lado motor:
            \begin{equation} \label{eq:motorIzaje}
                J_{hm+hb} \frac{d \omega_{hm}}{dt} = T_{hm}(t) + T_{hb}(t) - b_{hm} \omega_{hm}(t) - T_{hml}(t)
            \end{equation}

            relacion de transmision
            \begin{equation} \label{eq:transmisionIzaje}
                i_h = \frac{\omega_{hm}(t)}{\omega_{hd}(t)} = \frac{T_{hd}(t)}{T_{hml}(t)}
            \end{equation}

            si reemplazo \ref{eq:transmisionIzaje} en \ref{eq:motorIzaje} y despejo $T_{hd}(t)$

            \begin{equation} \label{eq:Thd}
                T_{hd}(t) = J_{hm+hb} \frac{d \omega_{hd}}{dt} {i_h}^2 - b_{hm} \omega_{hd}(t) {i_h}^2 + i_h (T_{hm}(t) + T_{hb}(t))
            \end{equation}

            reemplazando en \ref{eq:tamborIzaje} y operando se obtiene

            \begin{equation} \label{eq:izajeThdl}
                (J_{hd+hEb} + J_{hm+hb} i_h^2) \frac{d \omega_{hd}}{dt} = - (b_{hd} + b_{hm}i_h^2) \omega_{hd}(t) + i_h (T_{hm}(t) + T_{hb}(t)) + T_{hEb}(t) - T_{hdl}(t)
            \end{equation}

            como $T_{hdl}(t) = F_{hw}(t)*r_{hd}$, $2V_h = r_{hd}*\omega_{hd}(t)$ y $V_h = -\frac{dl_h(t)}{dt}$ y dividiendo por $r_{hd}$:

            \begin{equation} \label{eq:izajeFhw}
                2\frac{(J_{hd+hEb} + J_{hm+hb} i_h^2)}{r_{hd}^2} \frac{d^2 l_h(t)}{dt^2} = - 2\frac{(b_{hd} + b_{hm}i_h^2)}{r_{hd}^2} \frac{d l_h(t)}{dt} - \frac{i_h}{r_{hd}} (T_{hm}(t) + T_{hb}(t)) - \frac{T_{hEb}(t)}{r_{hd}} + F_{hw}(t)
            \end{equation}
            
            Reemplazando por parametros equivalentes:
            
            \begin{equation} \label{eq:izajeEquiv}
                M_{Eh} \ddot{l_h}(t) = - b_{Eh} \dot{l_h}(t) - \frac{i_h}{r_{hd}} (T_{hm}(t) + T_{hb}(t)) - \frac{T_{hEb}(t)}{r_{hd}} + F_{hw}(t)
            \end{equation}

            Donde

            \begin{align} \label{eq:izajeParamsEquiv}
                M_{Eh} = 2\frac{(J_{hd+hEb} + J_{hm+hb} i_h^2)}{r_{hd}^2}\\
                b_{Eh} = 2\frac{(b_{hd} + b_{hm}i_h^2)}{r_{hd}^2}\\
            \end{align}
            

        \subsubsection{Subsistema Carro}
            Segunda ley de Newton del lado tambor:
            \begin{equation} \label{eq:tamborCarro}
                J_{td} \frac{d \omega_{td}(t)}{dt} = T_{td}(t) - b_{td} \omega_{td}(t) - T_{tdl}(t)
            \end{equation}

            Segunda ley de Newton del lado motor:
            \begin{equation} \label{eq:motorCarro}
                J_{tm+tb} \frac{d \omega_{tm}(t)}{dt} = T_{tm}(t) + T_{tb}(t) - b_{tm} \omega_{tm}(t) - T_{tml}(t)
            \end{equation}

            relacion de transmision
            \begin{equation} \label{eq:transmisionCarro}
                i_t = \frac{\omega_{tm}(t)}{\omega_{td}(t)} = \frac{T_{td}(t)}{T_{tml}(t)}
            \end{equation}

            si reemplazo \ref{eq:transmisionCarro} en \ref{eq:motorCarro} y despejo $T_{td}(t)$

            \begin{equation} \label{eq:Ttd}
                T_{td}(t) = J_{tm+tb} \frac{d \omega_{td}(t)}{dt} {i_t}^2 - b_{tm} \omega_{td}(t) {i_t}^2 + i_t (T_{tm}(t) + T_{tb}(t))
            \end{equation}

            Reemplazo \ref{eq:Ttd} en \ref{eq:tamborCarro} y reordeno:

            \begin{equation} \label{eq:carroTtdl}
                (J_{td} + J_{tm+tb}*i_t^2) \frac{d \omega_{td}(t)}{dt} = i_t (T_{tm}(t) + T_{tb}(t)) - (b_{td} + b_{tm}{i_t}^2) \omega_{td}(t) - T_{tdl}(t)
            \end{equation}
        
            Como $\omega_{td}(t) r_{td} = V_{td}(t)$, $F_{tw}(t)r_{td} = T_{tdl}(t)$ y $V_{td}(t) = \frac{d x_{td}}{dt}$ y dividiendo por $r_{td}$:
            
            \begin{equation} \label{eq:carroFtw}
                \frac{(J_{td} + J_{tm+tb}*i_t^2)}{r_{td}^2} \frac{d^2 x_{td}(t)}{dt^2} = - \frac{(b_{td} + b_{tm}{i_t}^2)}{r_{td}^2} \frac{d x_{td}(t)}{dt} + \frac{i_t}{r_{td}} (T_{tm}(t) + T_{tb}(t)) - F_{tw}(t)
            \end{equation}

            Reemplazando por parametros equivalentes se obtiene la ecuacion del tambor del subsistema carro:

            \begin{equation} \label{eq:TamborCarro}
                M_{Etd} \ddot{x_{td}}(t) = - b_{Etd} \dot{x_{td}}(t) + \frac{i_t}{r_{td}} (T_{tm}(t) + T_{tb}(t)) - F_{tw}(t)
            \end{equation}

            La ecuacion de movimiento del carro es:

            \begin{equation} \label{eq:Carro}
                M_t \ddot{x_{t}}(t) = - b_t \dot{x_{t}}(t) + F_{tw}(t) + 2F_{hw}(t)\sin{\theta_l(t)}
            \end{equation}

            Y la fuerza transmitida por el cable del subsistema carro es:

            \begin{equation} \label{eq:fuerzaCableCarro}
                F_{tw}(t) = K_{tw}(x_{td}(t) - x_t(t)) + b_{tw}(\dot{x_{td}}(t) - \dot{x_t}(t))
            \end{equation}

            seria un sistema acoplado? preguntar si se resuelve asi

                
        \subsection{Diseño del controlador}

            \begin{equation}\label{eq:PID}
                T_m'(t) = b_ae_\omega(t) + K_{sa} e_\theta(t) + K_{sia}\int e_\theta(t) dt
            \end{equation}
            Por lo tanto, por Laplace:
            \begin{equation}\label{eq:PID_Laplace}
                T_m(s) = G(s)[b_aE_\omega(s) + K_{sa} \frac{1}{s} + K_{sia} \frac{1}{s^2}]E_\theta(s)
            \end{equation}

            Donde \(G_T(s)\) es la función de transferencia del modulador de torque que, como se supone ideal, es igual a 1.


            Para obtener la expresión que nos permita obtener las constante que definen al controlador se remplaza la ecuacion \ref{eq:PID} en la ecuacion de movimiento del izaje y del carro, se obtiene:
            Para el izaje, reemplazando \ref{eq:PID} en \ref{eq:izajeEquiv} y transformandola con Laplace, se obtiene:
            \begin{equation}\label{eq:izajeControl}
                M_{Eh} \ddot{L_h}(s) = - b_{Eh} sL_h(s) - \frac{i_h}{r_{hd}} [G(s)[b_aE_\omega(s) + K_{sa} \frac{1}{s} + K_{sia} \frac{1}{s^2}]E_\theta(s)] + F_{hw}(s)
            \end{equation}
            despejando 


            \subsubsection{Control de balanceo}
                Se deducen las ecuaciones de movimiento del sistema carro-péndulo, se obtiene:

                Planteando el equilibrio dinámico de los torques en el anclaje del péndulo:
                \begin{equation}
                    \sum \tau = I \ddot{\theta}
                \end{equation}
                \begin{equation}
                    m l^2 \ddot{\theta} = - m g \sin{\theta} + m \cos{\theta} \ddot{x}_t
                \end{equation}
                despejando \(\ddot{\theta}\):
                \begin{equation}
                    \ddot{\theta} = \frac{\cos{\theta} \ddot{x}_t}{l} - \frac{g \sin{\theta}}{l}
                \end{equation}

                También:
                \begin{equation}
                    x_l = \sin(\theta) l + x_t
                \end{equation}
                \begin{equation}
                    \dot{x}_l = \cos(\theta) \dot{\theta} l + \dot{x}_t
                \end{equation}

                Se definen el vector de estado como:
                \begin{equation}
                    x = \begin{bmatrix}
                        \theta \\
                        \dot{\theta}
                    \end{bmatrix}
                \end{equation}
                \begin{equation}
                    u = \ddot{x}_t
                \end{equation}
                \begin{equation}
                    y = \dot{x}_l
                \end{equation}

                Por lo tanto se expresa el modelo del sistema en el espacio de estados no lineal:
                \begin{equation}
                    \begin{cases}
                        \dot{x} = f(x,u,t) ; x(0) = x_0 \\
                        y = h(x,u,t)
                    \end{cases}
                \end{equation}
                Donde:
                \begin{equation}
                    f(x,u,t) = \begin{bmatrix}
                        \dot{\theta} \\
                        \frac{\cos{\theta} u}{l} - \frac{g \sin{\theta}}{l}
                    \end{bmatrix}
                \end{equation}
                \begin{equation}
                    h(x,u,t) = \cos{\theta} \dot{\theta} l
                \end{equation}
                Se ignora \(\dot{x}_t\) dado que buscaremos el incremendo de velocidad que debemos aplicar al carro para que el péndulo se mantenga en equilibrio.

                Se linealiza el sistema en torno a un punto de trabajo \(x(t), u(t)\) y se obtiene:
                \begin{equation}
                    \begin{cases}
                        \dot{x} = A x + B u \\
                        y = C x + D u
                    \end{cases}
                \end{equation}
                Donde:
                \begin{equation}
                    A_{ij}= \frac{\partial f_i}{\partial x_j} \Bigg|_{x(t), u(t)}
                \end{equation}
                \begin{equation}
                    B_{ij}= \frac{\partial f_i}{\partial u_j} \Bigg|_{x(t), u(t)}
                \end{equation}
                \begin{equation}
                    C_{ij}= \frac{\partial h_i}{\partial x_j} \Bigg|_{x(t), u(t)}
                \end{equation}
                \begin{equation}
                    D_{ij}= \frac{\partial h_i}{\partial u_j} \Bigg|_{x(t), u(t)}
                \end{equation}

                Se obtiene:
                \begin{equation}
                    A = \begin{bmatrix}
                        0 & 1 \\
                        -\cos{\theta}\frac{g}{l^2}-\sin{\theta}\frac{\ddot{x}_t}{l^2} & 0
                    \end{bmatrix}
                \end{equation}
                \begin{equation}
                    B = \begin{bmatrix}
                        0 \\
                        \frac{\cos{\theta}}{l^2}
                    \end{bmatrix}
                \end{equation}
                \begin{equation}
                    C = \begin{bmatrix}
                        -\sin{\theta}\dot{\theta}l & \cos{\theta}l
                    \end{bmatrix}
                \end{equation}
                \begin{equation}
                    D = 0
                \end{equation}

                Se propone un controlador PD para el sistema:
                \begin{equation}
                    u =  K_p (y^*-y) + K_d \frac{d}{dt}(y^*-y)
                \end{equation}

                \begin{equation}
                    \dot{x} = A x + B ( K_p (y^*-y) + K_d \frac{d}{dt}(y^*-y))
                \end{equation}
                \begin{equation}
                    \dot{x} = A x + B ( K_p (y^*-C x) + K_d \frac{d}{dt}(y^*-C x))
                \end{equation}

                \begin{equation}
                    \ddot{\theta} = A_{21} \dot{\theta} + B_{2} ( K_p (\dot{x}_l^*-C_1\theta-C_2\dot{\theta}) + K_d \frac{d}{dt}(\dot{x}_l^*-C_1\theta-C_2\dot{\theta}))
                \end{equation}
                \begin{equation}
                    \ddot{\theta} = A_{21} \theta + B_{2} K_p \dot{x}_l^* - B_{2} K_p C_1 \theta - B_{2} K_p C_2 \dot{\theta} + \frac{d}{dt} \left( B_{2} K_d \dot{x}_l^* - B_{2} K_d C_1 \theta - B_{2} K_d C_2 \dot{\theta} \right)
                \end{equation}
                Utilizando la transformada de Laplace:
                \begin{equation}
                    s^2 \Theta = A_{21} \Theta + B_{2} K_p \dot{X}_l^* - B_{2} K_p C_1 \Theta - B_{2} K_p C_2 s \Theta + B_{2} K_d \dot{X}_l^*s - B_{2} K_d C_1 \Theta s - B_{2} K_d C_2 \Theta s^2
                \end{equation}
                Despejando \(\theta/\dot{X}_l^*\):
                \begin{equation}
                    \Theta( s^2(B_{2} K_d C_2) + s (B_{2} K_p C_2 - B_{2} K_d C_1) - A_{21} + B_{2} K_p C_1) = \dot{X}_l^* (B_{2} K_p + B_{2} K_d s)
                \end{equation}

                \begin{equation} \label{eq:thetaXl}
                    \frac{\theta}{\dot{X}_l^*}= \frac{B_{2} K_p + B_{2} K_d s}{s^2(B_{2} K_d C_2) + s (B_{2} K_p C_2 - B_{2} K_d C_1) - A_{21} + B_{2} K_p C_1}
                \end{equation}

                Se obtinen las constantes \(K_p\) y \(K_d\) de forma que el denominado de \ref{eq:thetaXl} cumpla \(s^2 + s 2 \eta \omega + \omega^2 = 0\)

                \begin{equation}
                    \begin{cases}
                        2 \eta \omega = \frac{B_{2} K_p C_2 - B_{2} K_d C_1 }{B_{2} K_d C_2} \\
                        \omega^2 = \frac{A_{21} - B_{2} K_p C_1}{B_{2} K_d C_2}
                    \end{cases}
                \end{equation}
                \begin{equation}
                    \begin{cases}
                        2 \eta \omega = \frac{K_p}{K_d}-\frac{C_1 }{C_2} \\
                        \omega^2 = \frac{A_{21}}{B_{2} K_d C_2} - \frac{K_p C_1}{K_d C_2}
                    \end{cases}
                \end{equation}

                \begin{equation}
                    \begin{cases}
                        2 \eta \omega K_d=  K_p - K_d \frac{C_1 }{C_2} \\
                        \omega^2 K_d = \frac{A_{21}}{B_{2} C_2} - K_p \frac{C_1}{C_2}
                    \end{cases}
                \end{equation}

                \begin{equation}
                    \begin{cases}
                        K_p + K_d (-\frac{C_1}{C_2} - 2 \eta \omega) = 0 \\
                        K_p \frac{C_1}{C_2} + K_d \omega^2 = \frac{A_{21}}{B_{2} C_2} 
                    \end{cases}
                \end{equation}

                \begin{equation}
                    \begin{cases}
                        K_p = \frac{\frac{A_{21}}{B_{2} C_2}}{\omega^2-(-\frac{C_1}{C_2} - 2 \eta \omega)\frac{C_1}{C_2}} \\
                        K_d = \frac{-\frac{A_{21}}{B_{2} C_2} (-\frac{C_1}{C_2} - 2 \eta \omega)}{\omega^2-(-\frac{C_1}{C_2} - 2 \eta \omega)\frac{C_1}{C_2}}
                    \end{cases}
                \end{equation}

                \begin{equation}
                    \begin{cases}
                        K_p = \frac{\frac{A_{21}}{B_{2} C_2}}{\omega^2+(\frac{C_1}{C_2} + 2 \eta \omega)\frac{C_1}{C_2}} \\
                        K_d = \frac{\frac{A_{21}}{B_{2} C_2} (\frac{C_1}{C_2} + 2 \eta \omega)}{\omega^2+(\frac{C_1}{C_2} + 2 \eta \omega)\frac{C_1}{C_2}}
                    \end{cases}
                \end{equation}













            
                



            



\section{Resultados} \label{sec:results}

\section{Conclusión} \label{sec:conclusion}


% Estilo de citas
%\bibliographystyle{unsrt}

% Nombre del archivo .bib
%\bibliography{references}

\end{document}